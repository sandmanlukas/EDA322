\documentclass[a4paper,11pt]{article}
\usepackage{fancyhdr}
\usepackage[utf8]{inputenc}
%\setlength{\headheight}{11pt}

\lhead
\rhead
%\parskip 1em
%\parindent 0em

\begin{document}
\pagestyle{empty}
%-----------------------------------------------------------
\begin{titlepage}

\title{\Huge{Lab report} \\[0.1cm] \Large{Digital Design (EDA322)} \\ [0.4cm] \Large{ \emph{Writing Guidelines}} \\[0.4cm]}
\author{\large{\emph{Group TUE\_PM\_8}} \\[0.2cm] Niklas Gustafsson \\[0.05cm] Oskar Lundström \\[0.1cm]}
\maketitle
\thispagestyle{empty}
\end{titlepage}
\clearpage
%-----------------------------------------------------------
\pagestyle{fancyplain}
\pagenumbering{roman}
\tableofcontents
\clearpage
%%%%%%%%%%%%%%%%
% Introduction
\pagenumbering{arabic}
\setcounter{page}{1}
\section{Introduction}
(max: 1 page)
\\\\
This part will introduce the reader to the report. 

At the beginning, describe what the purpose of this lab report is. Then describe briefly what each section discusses and finally summarize the most important conclusions. 

\section{Method}
\subsection{Arithmetic and Logic Unit (ALU)}
(max: 2 pages)
\\\\
Describe what you did in lab 2 and what you have learnt. In addition, discuss your findings and observations during this lab. Summarize your answers to the questions in the lab PM and present the block diagrams that you have drawn. Furthermore, describe how your ALU performs subtraction using an adder. Remember to always explain your design choices and mention any assumptions. Finally, make use of figures and tables. 

\subsection{Top-level Design}
(max: 2 pages)
\\\\
Describe what you did in lab 3. In addition, describe how you implemented the bus using the mux and any extra logic or the tri-state buffers. Describe briefly how you implemented the storage elements that are used by the ChAcc processor. Show one snapshot of the simulation waveform where you write something to a memory location and then read from it. Remember to always explain your design choices and mention any assumptions. Finally, make use of figures and tables. 

\subsection{Controller}
(max: 2 pages)
\\\\
Describe what you did in lab4. More specifically, show the \emph{Finite-state machine} (FSM) of the controller by presenting the diagram you drew. Which design decisions did you make and why? Also include few waveforms, where you show that the controller runs correctly for some particular instructions using the provided testbench. Remember to always explain your design choices and mention any assumptions. Finally, make use of figures and tables. 

\subsection{Processor's Testbench}
(max: 2 pages)
\\\\
Describe what you did in lab5. More specifically, describe how you made the testbench to verify that your processor design was functionally correct. For example, you can specify how you generated inputs to the processor during the testing, how you were reading the expected outputs and how you compare the expected outputs with the actual outputs. Also mention if your processor design was working correctly from the beginning and if not describe how you backtrack the bugs. Remember to always explain your design choices and mention any assumptions. Finally, make use of figures and tables. 

\subsection{ChAcc on Nexys 3 board \emph{(Optional)}}
(max: 2 pages)
\\\\
Describe how you verified the correctness of your FPGA implementation. Note that the code that is executed on the implementation is the same code used for testing in Lab 5. You should compare sequences of values on various signals observed on the seven-segment displays to values seen in Modelsim simulation of the design. Please include in the report the sequence of program counter (PC) and display register values you observed during a successful execution on the FPGA. 

\subsection{Performance, Area and Power Analysis \emph{(Optional)}}
(max: 2 pages)
\\\\
To be announced in the Lab7PM.

\section{Analysis}
(max: 1 page)
\\\\
Summarize your results after performing all the labs (2, 3, 4 and 5).

Mention and discuss interesting findings and observations, as well as difficulties in completing some of the tasks of the four last labs.

After looking at your results, draw conclusions and describe briefly the learning outcome, that is what have you learnt by performing these labs?  

% Appendix
\newpage
\begin{appendix}

\section{Appendix}
(max: 4 pages)
\\\\
In the appendix, you can include extra figures or tables that don't fit in the main body of the lab report. 

\end{appendix}

\end{document}
